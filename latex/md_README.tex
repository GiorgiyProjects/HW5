infinite sparse matrix realization

Спроектировать 2-\/мерную разреженную бесконечную матрицу, заполненную значениями по умолчанию. Матрица должна хранить только занятые элементы -\/ значения которых хотя бы раз присваивались. Присвоение в ячейку значения по умолчанию освобождает ячейку. Необходимо уметь отвечать на вопрос -\/ сколько ячеек реально занято? Необходимо уметь проходить по всем занятым ячейкам. Порядок не имеет значения. Возвращается позиция ячейки и ее значение. При чтении элемента из свободной ячейки возвращать значение по умолчанию. Пример\+: // бесконечная матрица int заполнена значениями -\/1 \mbox{\hyperlink{classMatrix}{Matrix}}$<$int, -\/1$>$ matrix; assert(matrix.\+size() == 0); // все ячейки свободны auto a = matrix\mbox{[}0\mbox{]}\mbox{[}0\mbox{]}; assert(a == -\/1); assert(matrix.\+size() == 0); matrix\mbox{[}100\mbox{]}\mbox{[}100\mbox{]} = 314; assert(matrix\mbox{[}100\mbox{]}\mbox{[}100\mbox{]} == 314); assert(matrix.\+size() == 1); // выведется одна строка // 100100314 for(auto c\+: matrix) \{ int x; int y; int v; std\+::tie(x, y, v) = c; std\+::cout $<$$<$ x $<$$<$ y $<$$<$ v $<$$<$ std\+::endl; \} При запуске программы необходимо создать матрицу с пустым значением 0, заполнить главную диагональ матрицы (от \mbox{[}0,0\mbox{]} до \mbox{[}9,9\mbox{]}) значениями от 0 до 9. Второстепенную диагональ (от \mbox{[}0,9\mbox{]} до \mbox{[}9,0\mbox{]}) значениями от 9 до 0. Необходимо вывести фрагмент матрицы от \mbox{[}1,1\mbox{]} до \mbox{[}8,8\mbox{]}. Между столбцами пробел. Каждая строка матрицы на новой строке консоли. Вывести количество занятых ячеек. Вывести все занятые ячейки вместе со своими позициями. Опционально реализовать N-\/мерную матрицу. Опционально реализовать каноническую форму оператора {\ttfamily =}, допускающую выражения {\ttfamily ((matrix\mbox{[}100\mbox{]}\mbox{[}100\mbox{]} = 314) = 0) = 217} Самоконтроль
\begin{DoxyItemize}
\item индексация оператором {\ttfamily \mbox{[}\mbox{]}}
\item количество занятых ячеек должно быть 18
\item пакет {\ttfamily matrix} с бинарным файлом {\ttfamily matrix} опубликован на bintray Проверка Задание считается выполненным успешно, если после анализа кода, установки пакета и запуска приложения появился фрагмент матрицы, количество ячеек и список всех значений с позициями. 
\end{DoxyItemize}